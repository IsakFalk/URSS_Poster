% --------------------------------------------------------------------------- %
% Poster for the ECCS 2011 Conference about Elementary Dynamic Networks.      %
% --------------------------------------------------------------------------- %
% Created with Brian Amberg's LaTeX Poster Template. Please refer for the     %
% attached README.md file for the details how to compile with `pdflatex`.     %
% --------------------------------------------------------------------------- %
% $LastChangedDate:: 2011-09-11 10:57:12 +0200 (V, 11 szept. 2011)          $ %
% $LastChangedRevision:: 128                                                $ %
% $LastChangedBy:: rlegendi                                                 $ %
% $Id:: poster.tex 128 2011-09-11 08:57:12Z rlegendi                        $ %
% --------------------------------------------------------------------------- %
\documentclass[a0paper,portrait]{baposter}

\usepackage{relsize}		% For \smaller
\usepackage{url}			% For \url
\usepackage{epstopdf}	% Included EPS files automatically converted to PDF to include with pdflatex
\usepackage[font=small,labelfont=bf]{caption}
\usepackage{graphicx}
\usepackage{caption}
\usepackage{subcaption}
\usepackage{amsmath}


\renewcommand{\familydefault}{\sfdefault}

%%% Global Settings %%%%%%%%%%%%%%%%%%%%%%%%%%%%%%%%%%%%%%%%%%%%%%%%%%%%%%%%%%%

\graphicspath{{pix/}}	% Root directory of the pictures 
\tracingstats=2			% Enabled LaTeX logging with conditionals

%%% Color Definitions %%%%%%%%%%%%%%%%%%%%%%%%%%%%%%%%%%%%%%%%%%%%%%%%%%%%%%%%%
%% light blue backgroun, white background in the boxes
\definecolor{bordercol}{RGB}{40,40,40}
\definecolor{headercol1}{RGB}{186,215,230}
\definecolor{headercol2}{RGB}{80,80,80}
\definecolor{headerfontcol}{RGB}{0,0,0}
\definecolor{boxcolor}{RGB}{255,255,255}

%%%%%%%%%%%%%%%%%%%%%%%%%%%%%%%%%%%%%%%%%%%%%%%%%%%%%%%%%%%%%%%%%%%%%%%%%%%%%%%%
%%% Utility functions %%%%%%%%%%%%%%%%%%%%%%%%%%%%%%%%%%%%%%%%%%%%%%%%%%%%%%%%%%

%%% Save space in lists. Use this after the opening of the list %%%%%%%%%%%%%%%%
\newcommand{\compresslist}{
	\setlength{\itemsep}{1pt}
	\setlength{\parskip}{0pt}
	\setlength{\parsep}{0pt}
}

%%%%%%%%%%%%%%%%%%%%%%%%%%%%%%%%%%%%%%%%%%%%%%%%%%%%%%%%%%%%%%%%%%%%%%%%%%%%%%%
%%% Document Start %%%%%%%%%%%%%%%%%%%%%%%%%%%%%%%%%%%%%%%%%%%%%%%%%%%%%%%%%%%%
%%%%%%%%%%%%%%%%%%%%%%%%%%%%%%%%%%%%%%%%%%%%%%%%%%%%%%%%%%%%%%%%%%%%%%%%%%%%%%%

\begin{document}
\typeout{Poster rendering started}

%%% Setting Background Image %%%%%%%%%%%%%%%%%%%%%%%%%%%%%%%%%%%%%%%%%%%%%%%%%%
\background{
	\begin{tikzpicture}[remember picture,overlay]%
	\draw (current page.north west)+(-2em,2em) node[anchor=north west]
	{\includegraphics[height=1.1\textheight]{background}};
	\end{tikzpicture}
}

%%% General Poster Settings %%%%%%%%%%%%%%%%%%%%%%%%%%%%%%%%%%%%%%%%%%%%%%%%%%%
%%%%%% Eye Catcher, Title, Authors and University Images %%%%%%%%%%%%%%%%%%%%%%
\begin{poster}{
	grid=true,
	% Option is left on true though the eyecatcher is not used. The reason is
	% that we have a bit nicer looking title and author formatting in the headercol
	% this way
	%eyecatcher=false, 
	borderColor=bordercol,
	headerColorOne=headercol1,
	headerColorTwo=headercol2,
	headerFontColor=headerfontcol,
	% Only simple background color used, no shading, so boxColorTwo isn't necessary
	boxColorOne=boxcolor,
	headershape=roundedright,
	headerfont=\Large\sf\bf,
	textborder=rectangle,
	background=user,
	headerborder=open,
  boxshade=plain
}
%%% Eye Cacther %%%%%%%%%%%%%%%%%%%%%%%%%%%%%%%%%%%%%%%%%%%%%%%%%%%%%%%%%%%%%%%
{
	Eye Catcher, empty if option eyecatcher=false - unused
}
%%% Title %%%%%%%%%%%%%%%%%%%%%%%%%%%%%%%%%%%%%%%%%%%%%%%%%%%%%%%%%%%%%%%%%%%%%
{\sf\bf
	\begin{huge}
		A method for the estimation of distal dendro-dendritic gap-junctional parameters
	\end{huge}
}
%%% Authors %%%%%%%%%%%%%%%%%%%%%%%%%%%%%%%%%%%%%%%%%%%%%%%%%%%%%%%%%%%%%%%%%%%
{
	\vspace{0.5em} Isak Falk, Yulia Timofeeva\\
	{\smaller i.falk@warwick.ac.uk, y.timofeeva@warwick.ac.uk} \\
	{\smaller Complexity Science, University of Warwick, Gibbet Hill Road, Coventry, CV4 7AL} 
}
%%% Logo %%%%%%%%%%%%%%%%%%%%%%%%%%%%%%%%%%%%%%%%%%%%%%%%%%%%%%%%%%%%%%%%%%%%%%
{
% The logos are compressed a bit into a simple box to make them smaller on the result
% (Wasn't able to find any bigger of them.)
\setlength\fboxsep{0pt}
\setlength\fboxrule{0.5pt}
	\fbox{
		\begin{minipage}{22em}
			\includegraphics[width=11em,height=9.5em]{warwick_complexity_science.jpg}
			\includegraphics[width=11em,height=6.5em]{urss_logo.png}
		\end{minipage}
	}
}

%problem
\headerbox{Background}{name=background, column=0,row=0}{
Neurons are specialised cells which form the fundamental computing unit of the brain and the central nervous system. Each neuron consist of a cell body, dendrites and an axon, where the dendrites receive pulses of voltage from other neurons axons, which act like an output.

\begin{center}
	\includegraphics[width=0.49\linewidth]{neuron.png}
	\captionof{figure}[centering]{A neuron with an action potential going down the axon \cite{NeuronPic}}
\end{center}

Through voltage, the neurons may communicate to each other and this is what gives rise to the cognitive processes in any animal. When enough voltage enter a neuron, it spikes and send signals at a constant rate to all neurons connected to it, a so called action potential.

\begin{center}
	\includegraphics[width=0.49\linewidth]{neural_spike.jpeg}
	\captionof{figure}[centering]{Recording of membrane potential of a neuron \cite{spikes}}
\end{center}

On the level of a small scale network, or a single neuron, knowing the input/output relation when the cell membrane is subjected to an electrical current or spike lets us know a lot about the dynamics.

\\~\\

By finding a map between something easily measured, like voltage, and the strength and distance of a gap junction, it would be possible for experimentalists to recover the parameters of the gap junction, something that is hard to do directly. It is this question that I have considered in my research project.
}

\headerbox{References}{name=references,column=0,below=background}{
\smaller													% Make the whole text smaller
\vspace{-0.4em} 										% Save some space at the beginning
\bibliographystyle{plain}							% Use plain style
\renewcommand{\section}[2]{\vskip 0.05em}		% Omit "References" title
\begin{thebibliography}{1}							% Simple bibliography with widest label of 1
\itemsep=-0.01em										% Save space between the separation
\setlength{\baselineskip}{0.4em}					% Save space with longer lines
\bibitem{NeuronPic}
\bibitem{spikes}

\end{thebibliography}
}

\headerbox{Acknowledgement}
{name=Acknowledgement,span=1,column=0,below=references,above=bottom}{
I would like to thank the University of Warwick URSS for supporting me these weeks and my supervisor Yulia Timofeeva for guiding me and making this researh possible.

\begin{center}
	\includegraphics[width=\linewidth]{urss_full_logo.png}
\end{center}

}


%\cite{} sites work
%density
\headerbox{Method}{name=method,span=2,column=1,row=0}{

We use a model based on the cable equation modelling the voltage dynamics on the cell membranes of neurons. As the model is linear, we can specify the dynamics of the membrane-voltage completely by the so called Green's function $G_{ij}(x, y, t)$ which specifies how the voltage at length $x$ of branch $i$ develops in time with regards to a delta spike at length $y$ of branch $j$ at start. Throughout my project I only focused on the Green's function on the twin-cell network.

\begin{center}
	\includegraphics[width=0.25\linewidth]{two_cells-eps-converted-to.pdf}
	\captionof{figure}[centering]{Schema of twin-cell network}
\end{center}

To calculate the Green's function in the frequency domain we use the method of local point matching which depends on trips over the network from $x$ to $y$.

\vspace{-0.2em}
\begin{center}
	\includegraphics[width=0.36\linewidth]{model-eps-converted-to.pdf}
	\hspace{2em}
	\includegraphics[width=0.55\linewidth]{sum_rules-eps-converted-to.pdf}
\end{center}
\vspace{-0.2em}

The above figures show the type of nodes in the network and how the trips are modified by multiplication of constants depending on how they traverse the network going from $x$ to $y$. I looked at the dynamics of the network by plotting and analysing how the strength and distance of the gap-junction change the dynamics.

}

\headerbox{Results}
{name=Results,span=2,column=1,below=method}{
I calculated the response function for cell 1 and 2 in the symmetrical twin-cell network, input at the cell bodies and output at the cell body of cell 1 and different distances up until the gap junction for cell 2, the graphs show how the output (frequency domain) depends on distance and strength of gap junction for a specific distance of input from cell body.

\begin{center}
	\includegraphics[width=\linewidth]{cell1_2.jpg}
		\captionof{figure}[centering]{Graphs of cell 1 and cell 2, $L_{gj}$ is the distance and $g_{gj}$ the strength of the gap junction}
\end{center}

}

\headerbox{Conclusions}
{name=conclusions,span=2,column=1,below=Results,above=bottom}{



}

\end{poster}
\end{document}

